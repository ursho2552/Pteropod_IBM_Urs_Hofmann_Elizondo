%Global view on acidification
%transition to EBUS
%Transition to extreme events
%Indicator species
%method
%Results (none so far)


%IBP 2020 abstract version 3:
\noindent
\textbf{OLD ABSTRACT!}
Ocean acidification is projected to increase globally, with particularly severe consequences for marine calcifiers expected in the Eastern Boundary Upwelling Systems (EBUS), i.e. in the ocean’s most productive regions along the western boundaries of the continents. In these environments, marine calcifiers, such as the marine zooplankton group of the shelled pteropods (thecosomata), are naturally exposed to recurring acidification extreme events. However, the long-term increase in global ocean acidification and variability of upwelling-inducing winds will likely increase the magnitude, duration and recurrence of acidification extremes, and thus greatly enhance the impact of ocean acidification on these organisms. This study aims to quantify the population-level response of pteropods to changes in the acidification extremes in one of the EBUS, i.e. in the California Current System. To this end we developed an individual-based model of the pteropods to simulate their life cycle, diurnal vertical migration, and response to acidification extreme events of different magnitude and occurrence in a regional hindcast simulation (1979-2016) of the California Current System as a representative of the EBUS. The model was parameterized using compiled published experimental data of pteropod development, behaviour and response to acidification, and validated using published data on pteropod abundance and vertical distribution




%IBP 2020 abstract version 2:
% The ocean’s most productive and biogeochemically active regions, i.e. the Eastern Boundary Upwelling Systems (EBUS), are characterized by recurring acidification extreme events caused by the seasonal upwelling of corrosive waters through natural processes. The magnitude, duration, and occurrence timing of such events are affected by the long-term increase in variability of upwelling-inducing winds and global ocean acidification. Thus, in the EBUS the damage of marine calcifiers caused by extreme acidification events is enhanced compared to other regions. In this study, we aim to quantify the population-level response of marine calcifiers to the occurrence timing and magnitude of acidification extreme events in the EBUS. To this end, we developed an individual-based model to simulate the life cycle, diurnal vertical migration, and response to acidification extreme events of different magnitude and occurrence timing of thecosomatous (shelled) pteropods as model organisms for marine calcifiers in a regional hindcast simulation (1979-2016) of the California Current System as a representative of the EBUS. The model was parameterized using compiled published experimental data of shelled pteropod development, behaviour and response to acidification, and validated using published data on pteropod monthly abundance and field observations of their vertical distribution.


%IBP 2020 abstract version 1:
% Ocean acidification is projected to increase globally, where its consequences on marine calcifiers are likely enhanced in the ocean's most productive and biogeochemically active regions, i.e. the Eastern Boundary Upwelling Systems (EBUS). The EBUS are characterized by a seasonal and naturally recurring upwelling of corrosive waters, which superimposed on the long-term, global increase in ocean acidification generates acidification extreme events. The effects of such extreme events on marine ecosystems have previously been quantified using thecosomatous (shelled) pteropods, since they respond particularly sensitive to changes in ocean acidification, are well documented, represent a key component of the marine food-web, and dominate the carbon export among calcifiers in the open ocean. The aim of this study is to identify possible pteropod population bottlenecks caused by acidification extreme events using an individual-based model to simulate the life cycle of pteropods, their diurnal vertical migration, and response to acidification extreme events of different magnitude and occurrence timing in the California Current System. The model was parameterized using compiled published experimental data of shelled pteropod species development, behaviour and response to acidification, and validated using published data on pteropod monthly abundance and field observations of their vertical distribution. 


% The characteristics of acidification extreme events, such as the magnitude and timing, likely determine  

% Such acidification extreme events may expected to particularly impact marine calcifiers, where the occurrence timing of each extreme events 

% We investigate possible population bottlenecks caused by acidification extreme events using an individual-based model to simulate the population dynamics of shelled pteropods in the California Current System. Pteropods are considered indicator species for acidification, since they are of global interest, well-studied, and sensitive to changes in ocean acidification. The model was parametrized based on compiled published experimental information about shelled pteropod species behavior and development, and validated using published data from dispersion and trapping studies.


% The consequences of the global increase in ocean acidification are expected to be most severe for marine organisms in the Eastern Boundary Upwelling Systems (EBUS), i.e. in the ocean's most productive and biogeochemically active regions. These regions are characterized by a high natural and seasonally recurring variability, which regionally enhances ocean acidification and results in acidification extreme events. The long-term increase in ocean acidification, and acidification extremes are expected to particularly impact marine calcifiers. We investigate possible population bottlenecks caused by acidification extreme events using an individual-based model to simulate the population dynamics of shelled pteropods in the California Current System. Pteropods are considered indicator species for acidification, since they are of global interest, well-studied, and sensitive to changes in ocean acidification. The model was parametrized based on compiled published experimental information about shelled pteropod species behavior and development, and validated using published data from dispersion and trapping studies.








% The Eastern Boundary Upwelling Systems (EBUS) are among the most productive and biogeochemically active regions on the planet. They are regularly exposed to extreme events, are biodiversity hotspots for lower and higher marine trophic levels, with high phytoplankton and zooplankton abundances, which have rapid response times to climatic changes. 



% Organisms found in the EBUS are adapted to a high disturbance frequency, where a response to an extreme events is likely to be observed, since they live near their upper habitat as they are adapted to the high disturbance frequency, and some of them live near their upper habitat threshold, a response after an extreme events is likely to be observed \citep[e.g.][]{wishner2018ocean,bednarvsek2014limacina}. Thus using study regions with these criteria  allows the anticipation and assessment of potential ecosystem changes \citep{brotherton2015extreme}.


% Among the most productive areas are EBUS. In the Califorinia Current System we investigate the population level impacts of ocean acidification extreme events on marine calcifiers. Using an individual-based model we simulate the population dynamics of shelled pteropods in the California Current System.
% The model was parametrized based on compiled published experimental information about shelled pteropod species behavior and development, and validated using published data from dispersion and trapping studies

% and investigate possible population bottlenecks caused by acidification extreme events