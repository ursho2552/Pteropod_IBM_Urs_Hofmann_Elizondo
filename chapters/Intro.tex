%paragraph on ocean acidification

%paragraph on effects on marine ecosystems

%paragraph on pteropods

%Difficulties

%Goals





Euthecosomatous pteropods, i.e. shelled pteropods, are cosmopolitan holoplanktonic marine gastropods \citep{Bednarsek2012PteropodDistribution,SuarezMorales2009Gastropods} that represent major sources of food for higher trophic levels such as other zooplankton, fishes, seabirds, and whales \citep[][]{FalkPetersen2001Trophic,Hunt2008TopPredators}.
Among the fishes, commercially important ones, such as herring, mackarel, or salmon have been reported to consume large volumes of shelled pteropods \citep{LeBrasseur1966FishPteropods,lalli1989pelagic}.
In addition to their importance for commercial fisheries, shelled pteropods likely dominate the carbon export among marine calcifiers \citep{Buitenhuis2019CalciteExportPteropods}. Shelled pteropods are a vector for the export of particulate organic carbon to deeper waters due to the ballasting effect of their calcium carbonate shells \citep{Klaas2002Ballast}, the repackaging of buoyant particles into feacal pellets \citep{Treguer2003FecalPellets,Manno2009FeacalPellets}, and the production of mucus webs during feeding \citep{Gilmer1972FeedingBehaviour}, which aggregate and sediment particles that would not sink otherwise \citep{Noji1997Aggregation}.

Shelled pteropods are well documented organism \citep{Bednarsek2017ApplicationPteropodShell}, not only due to their ubiquity, commercial and biogeochemical importance mentioned above, but due to their sensitivity to ocean acidification \citep{Bednarsek2017ApplicationPteropodShell,Manno2017ReviewPteropodVulnerability}.
Euthecosomatous pteropods produce thin shells made out of the metastable calcium carbonate mineral aragonite \citep{Mucci1983Metastable,lalli1989pelagic}, which makes them particularly sensitive to changes in ocean chemistry \citep[][]{Orr2005Acidification,Bednarsek2014LimacinaHelicina}.
Thus, as shelled pteropods are among the first organisms to be affected by acidification, they are seen as an indicator for the overall response of marine ecosystems to ocean acidification \citep{Bednarsek2017ApplicationPteropodShell}. 

Ocean acidification decreases the overall fitness of the pteropod population. The exposure to acidified water conditions decreases shell growth \citep{Comeau2010RepairRates,Bednarsek2014CalcificationDissolution}, calcification \citep{Moya2016NervousSystem}, egg organogenesis \citep{Manno2016EggsAcidification}, sinking velocities \citep{Bednarsek2014CalcificationDissolution,Bergan2017SwimmingSinkingSpeeds}, mechanical protection against predators \citep{lalli1989pelagic,Comeau2010Predation}. Additionally, it may result in developmental delays \citep{Thabet2015Lifestages}, shell malformations, or even the complete loss of the aragonite shells \citep{Comeau2010Predation}. Overall, exposure to acidification reduces the survival probability of pteropods \citep{Bednarsek2017ApplicationPteropodShell}. Additionally, the developmental delay and impaired growth rate can reduce the carbon export to deeper layers due to the diminished ballasting effect \citep{Klaas2002Ballast,Bednarsek2014CalcificationDissolution}, this reduction in size may also decrease the flow of energy and carbon across trophic levels \citep{Capuzzo2017AbundanceRecruitment}, which ultimately diminishes global human protein resources \citep{Chassot2010Protein}. Thus, the consequences of exposure of shelled pteropods to acidification has severe implications for biogeochemical cycles and human health. 

In addition to the long-term increase in ocean acidification, shelled pteropods are simultaneously exposed to other environmental stressors, such as ocean warming, deoxygenation, or water freshening \citep{Bednarsek2016CumulativeEffects,Maas2012DVM,Manno2012SwimmingBehaviour}. Currently, some exposure studies \citep[e.g. ][]{Comeau2010RepairRates,Lischka2011WarmingAcidificationJuveniles,Lischka2012SynergisticEffects} have shown constrasting effects of multiple stressors on shelled pteropods. This uncertainty might be related to the species-specific adaptations/behaviours such as the diurnal vertical migration (DVM) or metabolic suppression \citep[DVM; ][]{Maas2012DVM}, life-stage specific sensitivities \citep{Kroeker2013JuvenilesCalcifiers,Manno2017ReviewPteropodVulnerability,Bednarsek2019MetaAnalysis}, or the exposure history of the pteropods \citep{Bednarsek2017ExposureHistory}. However, the combined effects of acidification, warming, and deoxygenation on shelled pteropods are likely cumulative in nature \citep{Bednarsek2016CumulativeEffects}.  For instance, co-occurring warming and acidification has been shown to increase pteropod mortality, shell degradation, and metabolic rate \citep{Lischka2012SynergisticEffects,Gardner2017SouthernOceanPteropods}. Another example is the combination of water freshening and acidification, where only the simultaneous exposure to acidification and reduced salinity decrease the ability of pteropods to swim upwards and increased their mortality \citep{Manno2012SwimmingBehaviour}. An increase in the metabolic rate, for instance due to warming, shell repair \citep{Hoshijima2017MetabolicRates}, or increased upward swimming activity \citep{Manno2012SwimmingBehaviour},  intensifies the pteropod oxygen requirements, which decreases their metabolic suppression strategy to cope with short-term deoxygenation and acidification \citep{Fabry2008MetabolicSuppression}. As the co-occurrence of ocean acidification, warming and deoxygenation is projected to increase globally \citep{Gruber2011TripleWhammy,?}, we can expect an intensification of readily observed effects of multiple stressors on shelled pteropods and marine ecosystems (?).


Currently, as explained by \citet{Manno2017ReviewPteropodVulnerability}, most stressor exposure experiments focus on the response of pteropods on shorter timescales, which might be linked to the difficulty to culture pteropods in a laboratory \citep{Howes2014Lab}, as well as the variable characterization of the life-cycle and longevity depending on the species and local environmental conditions \citep{Bednarsek2012PteropodDistribution,Manno2017ReviewPteropodVulnerability}. However, the focus of exposure research should take into account the life-cycle, and different generations of the shelled pteropods \citep{Manno2017ReviewPteropodVulnerability}, since juvenile marine calcifiers \citep{Kroeker2013JuvenilesCalcifiers}, and specifically juvenile pteropods \citep{Bednarsek2016CumulativeEffects}, have been identified as potential population bottlenecks. Recent studies have successfully documented the life-cycle of shelled pteropods with a clear distinction between life-stages \citep{Howes2014Lab,Thabet2015Lifestages}, as well as their longevity and population level growth rates \citep{Wang2017Lifecycle}. Such advances in our understanding of the shelled pteropods, allow us to simulate pteropod populations as a set of discrete individuals, i.e. using an Individual-Based Model \citep[IBM; ][]{DeAngelis2014IBM}. 


IBMs are a widely used tool to infer the impacts and responses of a changing environment on populations or communities on a regional or global scale \citep{DeAngelis2014IBM}. The organism chosen for the IBM can vary between a wide group \citep[e.g. phytoplankton; ][]{Clark2011IBMAdaptations} or a specific species \citep[e.g. Calanus finmarchicus, three-spined stickleback][]{Miller1998CalanusIBM,Mintram2018IBM_Stickleback}. Due to the species-specific dependence of pteropod growth rates and longevity to environmental conditions  \citep[e.g. temperature, food availability; ][]{Wang2017Lifecycle}, we focus herein on the group of shelled pteropods instead of individual species. This allows us include broad individual-level processes and behaviours \citep{DeAngelis2014IBM}, such as changes in acidification sensitivity along the life-cycle of the pteropods \citep{Bednarsek2016CumulativeEffects}, their calcification rate with changing shell size \citep{Bednarsek2014CalcificationDissolution}, their exposure to acidification due to DVM with increasing size \citep{Maas2012DVM,Bednarsek2015VerticalDistribution}, as well as changes in their abundance or shell dissolution and size across several generations. Additionally, in this study we focus on the pteropod population found in the Eastern Boundary Upwelling Systems, specifically the California Current System (CalCS), as a case study for our pteropod IBM, since changes in acidification (as well as warming and deoxygenation) are likely to disproportionately affect marine ecosystems in these regions \citep{Gruber2011TripleWhammy,?}. 





% Due to the dependence of pteropod growth rates and longevity on temperature \citep{Wang2017Lifecycle}, 



% As done for other marine organisms, e.g. marine phytoplankton \citep{Clark2011IBMAdaptations}, three-spined stickleback \citep{Mintram2018IBM_Stickleback}, Calanus finmarchicus \citep{Miller1998CalanusIBM}, or the Medfly \citep{Lux2018IBMAplication}, we present a shelled pteropod individual model which was parameterized using compiled publish experimental data of pteropod development, behaviour and their response to acidification. 

% are a widely used tool to infer the impacts and response of changing environmental conditions on populations or communities, and more importantly, they allow us to simulate the variation of individuals during 


% , such as the marine phytoplankton \citep{Clark2011IBMAdaptations}, three-spined stickleback \citep{Mintram2018IBM_Stickleback}. Additionally, they  Calanus finmarchicus \citep{Miller1998CalanusIBM}, Medfly \citep{Lux2018IBMAplication}. Despite the uncertainty in the growth rates and life-cycle of the shelled pteropods, they represent an optimal candidate for an IBM. As previously mentioned, shelled pteropods are cosmopolitan, and their geographic and vertical extent have been well-documented. Additionally, shelled pteropods have been shown to accumulate their exposure throughout their life, which with known calcification and dissolution rates allows us to trace....




% Goals of study...







% shell the thin aragonite shells of these pteropods , such as the long-term increased in temperature \citep{Lischka2011WarmingAcidificationJuveniles,Gardner2017SouthernOceanPteropods}, acidification \citep[][]{Bednarsek2017ApplicationPteropodShell,Manno2017ReviewPteropodVulnerability}, deoxygenation \citep[][]{Seibel2016HypoxiaMetabilic,Wishner2018DeoxygenetionThresholds}, or their combination \citep{Lischka2012SynergisticEffects,Bednarsek2016CumulativeEffects}


% of the shelled pteropods, the 
% These organisms are sensitive to changing environmental conditions caused by climate change, such as the long-term increased in temperature, acidification, and deoxygenation (REFS). The effects of rising temperature and declining oxygen levels on the metabolic rate (REFS), shell sizes (REFS), or fecundity (REFS) of pteropods has been documented. However, shelled pteropods are overwhelmingly used to quantify the effects of the increasing trend in marine acidification in laboratory conditions (REFS), as well as in several regions of the world such as the California Current System (REFS), Benguela current (REFS), Southern Ocean (REFS), North Pacific Ocean (REFS), Mediterranean Sea (REFS), among others (REFS). The use of shelled pteropods as indicators for acidification This is especially the case, since shelled pteropods produce a shell made out of the metastable carbonate mineral aragonite (REF). This metastable carbonate is more susceptible to increased acidification, and thus shelled pteropods are considered as early indicators for marine acidification (REFS).

% \textbf{EXTREMELY ROUGH PARAGRAPHS BELOW!}

% Temp sensitivity: Seibel et al., 2007; Comeau et al., 2010; Maas et al., 2011; Seibel et al., 2012)
% Marine acidification leads to the wide-spread dissolution of the shells of pteropods. This shell dissolution has been found to have far-reaching consequences for the pteropods themselves, the carbonate export, as well as for higher trophic levels. The dissolution of pteropod shells leads to the loss of buoyancy (REF), reduced sinking and swimming velocity (Bergan2017), developmental delays (REF), reduced shell sizes (REFS), reduced egg organogenesis (REF), reduced protection against predators (REF), and thus an overall fitness loss of the pteropod population. The effects of acidification on the shell size due to developmental delays or impaired growth rate reduces the carbonate export to deeper layers due to the diminished ballasting effect (bednarvsek2019). Additionally, the reduction in size and metabolic rate caused by acidification decreases the flow of energy and carbon across trophic levels (Capuzzo2017), and thus potentially diminishes global human protein resources (Chassot et al., 2010).

% With the projected increase in global marine acidification it is
% From laboratory experiments, it has been shown that the negative response of pteropods to corrosive waters varies with life-stages, where pteropod larvae and juveniles are affected more by higher acidification compared to eggs and adult pteropods (bednarsek2019?). Additionally, in some regions of the world, pteropods are adapted to cope with changes in acidification  the marine organisms in the EBUS are adapted to cope with changes in acidification, some of them live near their upper habitat threshold \citep{Wishner2018,Bednarsek2014}....
% While the effects of increasing acidification have been quantified, the distinction between long term acidification increases and acidification events has not been done. Shelled pteropods are regularly exposed to large changes in acidification due to their Diurnal Vertical Migration (DVM) \citep{ref}. During DVM, adult pteropods can travel up to $XX$ m and, for instance in the EBUS, they experience large changes in the aragonite saturation state \citep{REF}(THIS NEEDS TO BE INTRODUCED ABOVE). In general, the exposure to a reduced aragonite saturation state leads to shell dissolution in pteropods \citep{bednarsek}, loss of buoyancy \citep{REF}, reduced protection from predators \citep{REF}, developmental delays \citep{REF}, and thus a loss of their overall fitness. Additionally,  the negative response of pteropods to corrosive waters varies with life-stage, where pteropod larvae and juveniles are affected more by higher acidification compared to eggs and adult pteropods \citep{bednarsek}. Thus, acidification extremes can have deleterious effects on key life-stages of pteropods depending on their occurrence timing.


% \begin{itemize}
%     \item Introduction of shelled pteropods as keystone species for biogeochemical processes, and marine food-web.
%     \item Sensitivity of pteropods to climate change (temperature, oxygen, acidification) and using them to quantify changes in ecosystem health due to climate change (long-term trend).
%     \item Transition to global acidification and recorded effects on pteropods
%     \item Knowledge gaps on the effects of acidification on pteropods. Susceptibility of different life-stages, role of DVM, effect on abundance peaks, population size, and CaCO$_3$ content. 
%     \item Importance of filling these gaps with a focus on EBUS, since they represent the most productive regions of Earth, and the disproportionate impact of climate change in these regions.
%     \item Pteropod Individual-based model as a possible approach to fill gaps. Using examples from other IBMs (calanus, forams, phyto, etc)
% \end{itemize}


%  In general, the exposure to a reduced aragonite saturation state leads to shell dissolution in pteropods \citep{bednarsek}, loss of buoyancy \citep{REF}, reduced protection from predators \citep{REF}, developmental delays \citep{REF}, and thus a loss of their overall fitness. Additionally,  the negative response of pteropods to corrosive waters varies with life-stage, where pteropod larvae and juveniles are affected more by higher acidification compared to eggs and adult pteropods \citep{bednarsek}. Thus, acidification extremes can have deleterious effects on key life-stages of pteropods depending on their occurrence timing.












% The increasing acidification trend will disproportionately  affect marine calcifiers in the Eastern Boundary Upwelling Systems (EBUS), which are among the most productive and biogeochemically active regions of the planet \citep{chavez2009}. The uneven effects on marine calcifiers in the EBUS are caused by the combination of the increasing trend in global acidification, and the characteristic and naturally recurring upwelling of corrosive waters in these regions\citep{chan2008,hauri2009}, which results in acidification extreme events, i.e. the occurrence of corrosive waters in places where they are usually not found. While the marine organisms in the EBUS are adapted to cope with changes in acidification, some of them live near their upper habitat threshold \citep{Wishner2018,Bednarsek2014}, where acidification extremes will likely be observed.

% The effects of the increasing trend in marine acidification has been readily quantified for in several regions of the world, such as the California Current System, Benguela current, Southern Ocean, North Pacific Ocean, Mediterranean Sea, OTHER EXAMPLES WITH REFERENCES. These studies use the planktonic shelled pteropods to quantify the effect of ocean acidification on marine calcifiers. These organisms represent an optimal indicator species for the quantification of the effects of acidification, since they are cosmopolitan, are sensitive to changes in acidification, are well documented \citep{bednarsek2017b}, dominate the carbon export among pelagic calcifiers \citep{buithenhuis2019}, and are shown to lack an acclimatization capacity to prolonged acidification exposure \citep{bednarsek2017a}. 

% While the effects of increasing acidification have been quantified, the distinction between long term acidification increases and acidification events has not been done. Shelled pteropods are regularly exposed to large changes in acidification due to their Diurnal Vertical Migration (DVM) \citep{ref}. During DVM, adult pteropods can travel up to $XX$ m and, for instance in the EBUS, they experience large changes in the aragonite saturation state \citep{REF}(THIS NEEDS TO BE INTRODUCED ABOVE). In general, the exposure to a reduced aragonite saturation state leads to shell dissolution in pteropods \citep{bednarsek}, loss of buoyancy \citep{REF}, reduced protection from predators \citep{REF}, developmental delays \citep{REF}, and thus a loss of their overall fitness. Additionally,  the negative response of pteropods to corrosive waters varies with life-stage, where pteropod larvae and juveniles are affected more by higher acidification compared to eggs and adult pteropods \citep{bednarsek}. Thus, acidification extremes can have deleterious effects on key life-stages of pteropods depending on their occurrence timing.


% In this study we aim to quantify the effect of acidification events on the fitness of shelled pteropods in the CalCS. FURTHER DESCRIPTION HERE.

% Portable tomographic PIV measurements of swimming shelled Antarctic pteropods


% Title of the planned paragraphs:
% \begin{itemize}
%     \item The global long-term increase in ocean acidification and general effects on marine ecosystems.
%     \item Pronounced impacts of ocean acidification in Easter Boundary Upwelling Systems through extreme acidification.
%     \item Pteropods as a model organism for quantifying the impacts of acidification on marine ecosystems.
%     \item Are there potential gaps in this approach (e.g. population dynamics, life-stages, and cycle of pteropods) and why are they important? (Focus on different responses depending on life-stage, where the timing and magnitude of acidification exposure/extreme events becomes more important/damaging)
%     \item Goals: \begin{enumerate}
%         \item How has the timing and magnitude of acidification extremes changed since 1979-2016?
%         \item Does the population life-cycle change significantly from year to year?
%         \item What changes can be observed in the pteropod population (e.g. reduced abundance, developmental delay, timing of peak abundance, maximum shell diameter)?
%         \item Is there a potential relation between pteropod population-level bottlenecks and extreme event magnitude and timing?
%     \end{enumerate}
% \end{itemize}




% the combination of the naturally recurring changes in acidification and the global increasing acidification trend

% and are thus adapted to these conditions \citep{chan2008} due to upwelling and to the ongoing global increase in acidification. Additionally, with the projected

% Additionally, organisms living in these regions are adapted tothe high disturbance frequency [Chan et al., 2008],  and as some of these organisms live neartheir upper habitat threshold, a response after an extreme events is likely to be observed [e.g.Wishner et al., 2018; Bednarˇsek et al., 2014].

% These regions, while only covering $XX\%$ of the global surface area, produce around $20\%$ of the global capture fisheries \citep{pauly1995}


% Increases in marine acidification disproportionally affect marine calcifiers in the most productive and biogeochemically active regions of the planet, i.e. in the Eastern Boundary Upwelling Systems (EBUS).




% Increased marine acidification disproportionally affects marine organisms in the Eastern Boundary Upwelling Systems (EBUS). These systems are among the most productive and biogeochemically active regions on the planet \citep{Chavez2009}. They cover $XX\%$ of the global surface area and produce around $20\%$ of the global capture fisheries \citep{pauly1995}.

% The  California Current System (CS) has become more acidic due to the uptake of anthropogenic CO$_2$ by the ocean \citep{feely2008}. The area of low pH in the CS is projected to increase and persist for the future \citep{Gruber2012}. The CS is among the most productive and biogeochemically active regions on the planet \citep{chavez2009}, it provides around $20\%$ of the global capture fisheries \citep{pauly}


% The EBUS areamong  the  most  productive  and  biogeochemically  active  regions  on  the  planet  [Chavez  andMessi ́e, 2009].  They provide around20$\%$
% of the global capture fisheries [Pauly and Christensen,1995].  In analogy to terrestrial case study regions, they (i) are regularly exposed to extremeevents [e.g.  acidification; Hauri et al., 2009],  and (ii) are biodiversity hotspots for lower andhigher trophic levels [Checkley and Barth, 2009], with (iii) high phytoplankton and zooplanktonabundances, which have rapid response times to climatic changes [Paerl et al., 2003; Kavanaughet al., 2016; HAYS et al., 2005].  Additionally, organisms living in these regions are adapted tothe high disturbance frequency [Chan et al., 2008],  and as some of these organisms live neartheir upper habitat threshold, a response after an extreme events is likely to be observed [e.g.Wishner et al., 2018; Bednarˇsek et al., 2014].


%  For in-stance,  the  uptake  of  anthropogenic  CO2by  the  ocean  increased  the  area  affected  by  lowerpH-waters in the California Current System [Feely et al., 2008], and this area is projected toincrease or persist for the future [Gruber et al., 2012].

% The uptake of anthropogenic CO$_2$ by the ocean has increased the area affected by lower pH-waters on both global and local scales \citep{feely2008}.


% Auf der Basis dieser Analysen wissen wir ja, wann und wo wir es mit einem Extremereignis zu tun haben. Damit können wir Organismen vergleichen, die eine erhöhte Exposure hatten zu denjenigen die selten in einem Extremereignis waren. 



% Effekt vor allem durch den langfristigen Trend herkommt, oder ob tatsächlich die Extremereignisse eine überproportionale Rolle spielen
