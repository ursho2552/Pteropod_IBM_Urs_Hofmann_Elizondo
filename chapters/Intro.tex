%New structure:

%Why are pteropods important?
Shelled pteropods or sea butterflies, are cosmopolitan marine sea snails that spend their entire life cycle as free-swimming organisms \citep{Bednarsek2012PteropodDistribution,SuarezMorales2009Gastropods}. Pteropods represent a  major food source for other zooplankton, seabirds, whales, and fish \citep[][]{FalkPetersen2001Trophic,Hunt2008TopPredators} including commercially important ones such as herring, mackerel, or salmon \citep{LeBrasseur1966FishPteropods,lalli1989pelagic}.
In addition pteropods likely dominate the carbon export among pelagic marine calcifiers \citep{Buitenhuis2019CalciteExportPteropods} through different processes. For instance, they act as a vector for the export of particulate organic carbon to deeper waters due to the ballasting effect of their shells \citep{Klaas2002Ballast}, they repackage buoyant particles into faecal pellets \citep{Treguer2003FecalPellets,Manno2009FeacalPellets}, produce mucus webs during feeding \citep{Gilmer1972FeedingBehaviour}, which aggregate, and sediment particles that would not sink otherwise \citep{Noji1997Aggregation}.


Pteropods represent an ideal sentinel species for the overall response of marine ecosystems to ocean acidification due to their aragonite shells \citep{Orr2005Acidification,Bednarsek2017ApplicationPteropodShell}. Pteropods produce thin shells made out of the metastable calcium carbonate mineral aragonite \citep{lalli1989pelagic}, which is approximately $50\%$ more soluble than its calcite counterpart \citep{Mucci1983Metastable}. Thus, a decrease in seawater carbonate concentration and pH due to the marine uptake  of  atmospheric $CO_2$ \citep{Doney2009OtherCO2} has been observed to lead  to  widespread  pteropod  shell  dissolution  and  a  decrease in their survival \citep{Wall_Palmer2013dissolution,Bednarsek2017ApplicationPteropodShell,Manno2017ReviewPteropodVulnerability}. In addition, the exposure to lower carbonate concentration and pH deccreases shell growth \citep{Comeau2010RepairRates,Bednarsek2014CalcificationDissolution}, calcification \citep{Moya2016NervousSystem}, egg organogenesis \citep{Manno2016EggsAcidification}, downward swimming velocities \citep{Bednarsek2014CalcificationDissolution,Bergan2017SwimmingSinkingSpeeds}, and mechanical protection against predators \citep{lalli1989pelagic,Comeau2010Predation}. In addition, the exposure to lower carbonate concentration is linked to developmental delays \citep{Thabet2015Lifestages}, shell malformations or complete loss of shells \citep{Comeau2010Predation}. The exposure to corrosive waters not only affects pteropods, but has far reaching consequences for biogeochemical cycles \citep[e.g. ][]{Bednarsek2014CalcificationDissolution} and the flow of energy across trophic levels \citep[e.g. ][]{Capuzzo2017AbundanceRecruitment}. For instance, developmental delays and impaired growth rates reduce the carbon export to deeper layers due to a diminished ballasting effect of pteropod shells \citep{Klaas2002Ballast,Bednarsek2014CalcificationDissolution}, or decrease the flow of energy and carbon across trophic levels \citep{Capuzzo2017AbundanceRecruitment}, which diminishes fisheries yield \citep{ware2005bottom} and global food resources \citep{Chassot2010Protein}. The impacts of ocean acidification on the survival of pteropods depend not only on the magnitude and duration of the exposure to corrosive waters, but on the life stage of the pteropods \citep{Bednarsek2016CumulativeEffects}, as well as their exposure history \citep{Bednarsek2017ExposureHistory}. 
%Thus, a spatio-temporal characterization of shell dissolution that includes the life cycle of pteropods over several generations, with life stage specific responses and sensitivities to ocean acidification likely provides an accurate picture of the effects of acidification on pteropods \citep{Johnson2016,Bednarsek2016CumulativeEffects,Manno2017ReviewPteropodVulnerability,Bednarsek2019MetaAnalysis}.


The response of pteropods to acidification is mainly assessed using short term incubation studies \citep[less than one month; ][]{Manno2017ReviewPteropodVulnerability,Doo2020Attribution}. However, the scalability of the response of individuals to pteropod populations might be limited \citep{Doo2020Attribution,Andersson2015LabPopulation}, since most studies focus on the response of adults rather than the vulnerable early life stages \citep{Ross2011early,Busch2016directEffects}, assume static ocean chemistry conditions \citep{Bednarsek2019MetaAnalysis}, and do not consider how acidification responses during early life stages might diminish the ability for pteropods to deal with future exposure to acidification \citep{Bednarsek2017ExposureHistory} or whether the response has an impact on population dynamics \citep{Andersson2015LabPopulation,Busch2015Uncertainties}.
%, which do not reflect the natural variability measured in the field \citep{Hofmann2011Variability}, 
In contrast, available in-situ time series studies, while having the highest degree of realism and ability to quantify population-level responses to acidification \citep{Andersson2015LabPopulation}, have shown no change \citep{Ohman2009Multi,Head2010NWAtlantic,Thibodeau2018WAP}, declining \citep{Beaugrand2012NAtlantic,Beare2013NorthSea,Mackas2011Timeseries}, and increasing \citep{Howes2015Mediterranean} trends in pteropod abundance around the world. In addition, a relationship between pteropod abundance and acidification could not be found in any of the aforementioned studies \citep{Doo2020Attribution}. This lack of relationship between pteropod abundance and acidification is likely linked to the variable responses of pteropods in nature \citep{Bednarsek2016CumulativeEffects}, or that the tipping point at which ocean acidification influences pteropods on a population level has not been crossed \citep{Mackas2011Timeseries,Doo2020Attribution}. However, the latter stands in contrast to the observed widespread shell dissolution \citep{Bednarsek2015VerticalDistribution}, suggesting that studies of abundance time series might have neglected indirect or sublethal effects that leave the abundance unaffected but influence shell structure, morphology, vertical distribution, or population demographics \citep{Ohman2009Multi,Bednarsek2015VerticalDistribution,Manno2017ReviewPteropodVulnerability,Bednarsek2019MetaAnalysis,Doo2020Attribution}. Thus, a spatio-temporal characterization of shell dissolution that includes the life cycle of pteropods over several generations, with life stage specific responses and sensitivities to ocean acidification  will likely provide an accurate picture of the effects of acidification on pteropod populations \citep{Johnson2016,Bednarsek2016CumulativeEffects,Manno2017ReviewPteropodVulnerability,Bednarsek2019MetaAnalysis}. 



%Life cycle/growth the problem and possible solution
A possible approach to resolve the aforementioned limitations is the simulation of pteropod populations as a set of discrete individuals over several generations using an Individual-Based Model \citep[IBM; ][]{DeAngelis2014IBM}. This approach requires a generalized life cycle of shelled pteropods and their growth rates. Despite the ubiquity, commercial and biogeochemical importance of pteropods, their life cycle and growth rates have not been fully described \citep{Hunt2008TopPredators,Manno2017ReviewPteropodVulnerability}, due to the difficulty to culture them in a laboratory \citep{Howes2014Lab}, and the dependence of their life cycle and longevity on local environmental conditions or species \citep{Bednarsek2012PteropodDistribution,Wang2017Lifecycle,Manno2017ReviewPteropodVulnerability}. However, unlike pteropods in the polar regions \citep[e.g. ][]{Kobayashi1974Growth,Gannefors2005Overwintering,Hunt2008TopPredators,Bednarsek2012Population}, in the temperate zone a life cycle with two generations has been reported across different regions and species, such as in the east coast of South America for \textit{L. retroversa australis}  \citep[][]{Dadon1992Reproduction}, the coast of Nova Scotia and the Gulf of Maine for \textit{L. retroversa} \citep[][]{lalli1989pelagic,Maas2020Lipids}, or the temperate North Pacific for \textit{L. helicina} \citep[][]{Wang2017Lifecycle}. The life cycle of temperate pteropods has an overwintering generation that grows slowly during winter, reaches a size between $1 \, mm$ and $3\, mm$, spawns a spring generation, and dies shortly after \citep{Dadon1992Reproduction,Wang2017Lifecycle,Maas2020Lipids}. The spring generation grows rapidly to maturity, reaches relatively smaller sizes ($0.5\, - \, 1 \, mm$) and dies shortly after spawning the next overwintering generation in late autumn \citep{Dadon1992Reproduction,Wang2017Lifecycle,Maas2020Lipids}. The life cycle and growth rates of pteropods in temperate regions are likely less influenced by seasonal changes in environmental conditions compared to the polar species \citep{Manno2017ReviewPteropodVulnerability}, and their longevity and growth rates are consistent across species and regions. Additionally, recent culturing experiments of the temperate \textit{L. retroversa's} life cycle identify ten stages and the developmental timings of key organs/traits \citep{Howes2014Lab,Thabet2015Lifestages}, which refines the description of the life cycle of temperate pteropods. 

The pteropod life cycle includes the following stages (and developmental timing after spawning): the 2- (4 hours), 4- (6 hours), 8- (9 hours), 16-cell (11 hours), blastula (16 hours), gastrula (24-72 hours), hatching (3 days), trochophore (3-6 days), veliger (6-7 days), non-reproductive juveniles (1 month), and reproductive adult (3 months) \citep{Thabet2015Lifestages}. The development of key organs include the larval shell (protoconch) and wings (parapodia) after six days from hatching, and three weeks from hatching, respectively \citep{Thabet2015Lifestages}.  This detailed life cycle description with the developmental timings of key organs is crucial for the interpretation of the response of pteropods to ocean acidification, since the development of the protoconch marks the onset of early shell dissolution \citep{Thabet2015Lifestages,Johnson2016}, which likely diminishes the ability for pteropods to deal with upcoming shell dissolution \citep{Bednarsek2017ExposureHistory}, and the development of parapodia marks the onset of the daily upward and downward migration of pteropods across the water column, i.e. diel vertical migration \citep[DVM; ][]{lalli1989pelagic,Mackas2005DVM,Hunt2008TopPredators}, which exposes pteropods to corrosive waters at deeper layers and non-corrosive ones near the surface \citep{Bednarsek2015VerticalDistribution}. An additional size or stage dependent trait is the calcification rate or shell repair rate \citep{Comeau2010RepairRates}, which increases with pteropod size \citep{Bednarsek2014CalcificationDissolution}, making juveniles performing DVM relatively more susceptible to shell dissolution than adult organism. Given the consensus in the life cycle of temperate pteropods across regions and species \citep{lalli1989pelagic,Dadon1992Reproduction,Wang2017Lifecycle,Maas2020Lipids}, and the recent success in culturing experiments detailing the stages and developmental timings \citep{Howes2014Lab,Thabet2015Lifestages}, a pteropod IBM can be implemented based on the life cycle of temperate pteropods.




% The reported life cycle and size of pteropods in the temperate regions stand in contrast to the high variability in the longevity and size of pteropods in the polar regions from field observations, e.g. \textit{Limacina helicina antarctica} is estimated to live between one \citep{Gannefors2005Overwintering,Hunt2008TopPredators} and three or more years \citep{Bednarsek2012Population}, and adult organisms can reach a diameter between $3.7 \, mm$ \citep{Kobayashi1974Growth} and $9.6\, mm$ \citep{Gannefors2005Overwintering}. Since the life cycle and growth rates of pteropods in temperate regions are likely less influenced by seasonal changes in environmental conditions compared to the polar species \citep{Manno2017ReviewPteropodVulnerability}, and their longevity and growth rates are consistent across species and regions, the reported two generation per year life cycle may be used as a generalized life cycle description for shelled pteropods.



%%% exapnd on this paragraph!!!
% The response of pteropods to acidification is mainly assessed using short term incubation studies \citep[less than one month; ][]{Manno2017ReviewPteropodVulnerability,Doo2020Attribution}. However, the scalability of the response of individuals to pteropod populations might be limited \citep{Doo2020Attribution,Andersson2015LabPopulation}, since most studies focus on the response of adults rather than the vulnerable early life stages \citep{Ross2011early,Busch2016directEffects}, assume static ocean chemistry conditions \citep{Bednarsek2019MetaAnalysis}, and do not consider how acidification responses during early life stages might diminish the ability for pteropods to deal with future exposure to acidification \citep{Bednarsek2017ExposureHistory} or whether the response has an impact on population dynamics \citep{Andersson2015LabPopulation,Busch2015Uncertainties}.
% %, which do not reflect the natural variability measured in the field \citep{Hofmann2011Variability}, 
% In contrast, available in-situ time series studies, while having the highest degree of realism and ability to quantify population-level responses to acidification \citep{Andersson2015LabPopulation}, have shown no change \citep{Ohman2009Multi,Head2010NWAtlantic,Thibodeau2018WAP}, declining \citep{Beaugrand2012NAtlantic,Beare2013NorthSea,Mackas2011Timeseries}, and increasing \citep{Howes2015Mediterranean} trends in pteropod abundance around the world. In addition, a relationship between pteropod abundance and acidification could not be found in any of the aforementioned studies \citep{Doo2020Attribution}. This lack of relationship between pteropod abundance and acidification is likely linked to the variable responses of pteropods in nature \citep{Bednarsek2016CumulativeEffects}, or that the tipping point at which ocean acidification influences pteropods on a population level has not been crossed \citep{Mackas2011Timeseries,Doo2020Attribution}. However, the latter stands in contrast to the observed widespread shell dissolution \citep{Bednarsek2015VerticalDistribution}, suggesting that studies of abundance time series might have neglected indirect or sublethal effects that leave the pteropod abundance unaffected but influence shell structure, morphology, vertical distribution, or population demographics \citep{Ohman2009Multi,Bednarsek2015VerticalDistribution,Manno2017ReviewPteropodVulnerability,Bednarsek2019MetaAnalysis,Doo2020Attribution}. A possible approach to resolve these limitations is the simulation of pteropod populations as a set of discrete individuals over several generations using an Individual-Based Model \citep[IBM; ][]{DeAngelis2014IBM} based on the generalized life cycle of shelled pteropods.



%IBM paragraph (Reduce/clarify paragraph. see MV comments 18/08/2020!!)
IBM's have been used successfully in marine applications to characterize the life history of planktonic organisms, or life stages and the interaction between individuals and their surroundings \citep[e.g. ]{Werner1997Fish,Parada2003Anchovy,Miller1998CalanusIBM}. IBMs are well suited to link parameterised responses of individuals to environmental drivers and predict the overall responses of populations \citep{Stillman2014BenefitsIBM}. 
%Additionally, unlike traditional differential equation population modeling approaches, 
In IBMs the population-level response is the sum over all individuals and their interactions between each other and their environment \citep{DeAngelis2014IBM}, and they are thus able to consider variations among individuals of a population, and along their life cycle \citep{DeAngelis2014IBM}. This accounts for the dynamic and heterogeneous environmental conditions that determine population-level characteristics, such as size or life stage structure, in time and space \citep{Stillman2014BenefitsIBM}. 
%The organism chosen for IBMs can vary between an encompassing group \citep[e.g. phytoplankton; ][]{Clark2011IBMAdaptations} or a specific species \citep[e.g. Calanus finmarchicus, three-spined stickleback, zebrafish; ][]{Miller1998CalanusIBM,Mintram2018IBM_Stickleback,Beaudouin2015Zebra}.
In the case of shelled pteropods, their response to ocean acidification has been shown to depend on their life stage \citep{Bednarsek2016CumulativeEffects}, life history \citep{Bednarsek2017ExposureHistory}, behaviour \citep{Bednarsek2015VerticalDistribution} and size \citep{Bednarsek2014CalcificationDissolution}. For instance, younger pteropods are more sensitive to corrosive waters relative to adult pteropods \citep{Bednarsek2016CumulativeEffects}, smaller organisms are predominantly found near the surface, while larger ones might actively avoid corrosive waters \citep{Bednarsek2015VerticalDistribution}, or the shell repair rates \citep{Bednarsek2014CalcificationDissolution}, allow individuals to compensate from the shell damage due to corrosive conditions \citep{Comeau2010RepairRates}. Thus, a pteropod IBM considers indirect or sublethal effects of acidification, and unlike traditional time series analyses, might be able to aid in finding population-level responses to acidification.

% As mentioned previously, traditional in-situ pteropod abundance time series analyses have not found a link to acidification \citep[e.g. ][]{Ohman2009Multi}, since they do not consider indirect or sublethal effects \citep{Bednarsek2015VerticalDistribution,Manno2017ReviewPteropodVulnerability,Bednarsek2019MetaAnalysis,Doo2020Attribution}. 


% Additionally, the population-level approach neglects the size, life stage, or behavioral variations within the pteropod population, where for instance younger pteropods are more sensitivity to corrosive waters relative to adult pteropods \citep{Bednarsek2016CumulativeEffects}, smaller organisms are predominantly found near the surface, while larger ones might actively avoid corrosive waters \citep{Bednarsek2015VerticalDistribution}, or the shell repair rate of pteropods \citep{Bednarsek2014CalcificationDissolution}, which might allow them to compensate from the shell damage due to corrosive conditions \citep{Comeau2010RepairRates}.  

% In-situ observations of the effects of changing ocean chemistry on shelled pteropods has shows widespread shell dissolution on an individual levelfor the most productive regions of the . However, 
In this study, we implement a shelled pteropod IBM using the life cycle with two generations of temperate pteropods \citep{lalli1989pelagic,Dadon1992Reproduction,Wang2017Lifecycle,Maas2020Lipids} and the developmental timings of their life stages \citep{Howes2014Lab,Thabet2015Lifestages}. In addition, we implement the acidification sensitivity along the life cycle of pteropods \citep{Bednarsek2016CumulativeEffects}, calcification rate with changing shell size \citep{Bednarsek2014CalcificationDissolution}, exposure to acidification due to a size dependent DVM \citep{Maas2012DVM,Bednarsek2015VerticalDistribution}, and swimming velocities \citep{Bergan2017SwimmingSinkingSpeeds}, to analyze the outcome of our IBM on the output of a circulation model for the California Current System (CalCS). This allows us to test the hypotheses that the pteropod IBM (i) reproduces the pteropod abundance signal measured in the CalCS, (ii) reproduces the measured size spectrum across seasons, and (iii) quantifies the spatio-temporal variability of shell dissolution in the CalCS throughout the life cycle of shelled pteropods.
%, and (iii) quantifies shell dissolution experienced during DVM and upwelling events.






% Population-level approaches have failed to take into account indirect or sublethal effects of environmental stressor for shelled pteropods, where in-situ observations show extensive shell dissolution due to changes in ocean chemistry. In order to quantify potential sublethal effects of changes in ocean chemistry on a population-level,  we first develop a temperate shelled pteropod IBM, which we coupled to a circulation model for the California Current System. This has been rendered possible by the multitude of well studied responses of pteropods to changes in ocean chemistry on an individual level. These responses include the changes in acidification sensitivity along the life-cycle of the pteropods \citep{Bednarsek2016CumulativeEffects}, calcification rate with changing shell size \citep{Bednarsek2014CalcificationDissolution}, exposure to acidification due to a size dependent DVM \citep{Maas2012DVM,Bednarsek2015VerticalDistribution}, or changes in swimming velocities \citep{Bergan2017SwimmingSinkingSpeeds}. The collective response of each individual pteropod is used to determine the sublethal effects of changes in acidification for the population. In this study we model a pteropod population in the Eastern Boundary Upwelling Systems, specifically the California Current System (CalCS), as a case study for our temperate pteropod IBM, since changes in acidification are likely to disproportionately affect marine ecosystems in these regions \citep{Gruber2011TripleWhammy}. Our IBM (i) reproduces the pteropod abundance signal measured in the CalCS, (ii) quantifies the spatio-temporal variability of shell dissolution in the CalCS throughout the life cycle of shelled pteropods, and (iii) quantifies shell dissolution experienced during DVM and upwelling events.



%How does this problem manifest in current studies
%The life cycle of shelled pteropods has remained a source of uncertainty for the quantification of their response to changes in ocean chemistry \citep{Bednarsek2016CumulativeEffects,Bednarsek2017ApplicationPteropodShell,Manno2017ReviewPteropodVulnerability,Bednarsek2019MetaAnalysis}. 
% Pteropods represent an ideal sentinel species for the overall response of marine ecosystems to ocean acidification due to their aragonite shells \citep{Orr2005Acidification,Bednarsek2017ApplicationPteropodShell}. Pteropods produce thin shells made out of the metastable calcium carbonate mineral aragonite \citep{lalli1989pelagic}, which is approximately $50\%$ more soluble than its calcite counterpart \citep{Mucci1983Metastable}. Thus, a decrease in seawater carbonate concentration and pH due to the marine uptake  of  atmospheric $CO_2$ \citep{Doney2009OtherCO2} has been observed to lead  to  widespread  pteropod  shell  dissolution  and  a  decrease in their survival \citep{Wall_Palmer2013dissolution,Bednarsek2017ApplicationPteropodShell,Manno2017ReviewPteropodVulnerability}. In addition, the exposure to lower carbonate concentration and pH are linked to decreased shell growth \citep{Comeau2010RepairRates,Bednarsek2014CalcificationDissolution}, calcification \citep{Moya2016NervousSystem}, egg organogenesis \citep{Manno2016EggsAcidification}, downward swimming velocities \citep{Bednarsek2014CalcificationDissolution,Bergan2017SwimmingSinkingSpeeds}, mechanical protection against predators \citep{lalli1989pelagic,Comeau2010Predation}, as well as developmental delays \citep{Thabet2015Lifestages}, shell malformations or complete loss of shells \citep{Comeau2010Predation}. The exposure to corrosive waters not only affects pteropods, but is linked to far reaching consequences for the biogeochemical cycles and food webs. For instance, developmental delays and impaired growth rates reduce the carbon export to deeper layers due to a diminished ballasting effect \citep{Klaas2002Ballast,Bednarsek2014CalcificationDissolution}, and decrease the flow of energy and carbon across trophic levels \citep{Capuzzo2017AbundanceRecruitment}, which ultimately diminishes global human protein resources \citep{Chassot2010Protein}. Furthermore, the impacts of acidification on the survival of pteropods depend not only on the magnitude and duration of the exposure to corrosive waters, but on the life stage of the pteropods \citep{Bednarsek2016CumulativeEffects}, as well as their exposure history \citep{Bednarsek2017ExposureHistory}.  Thus, the spatio-temporal characterization of the shell dissolution needs to be considered in context with the life cycle of pteropods over several generations, where life stage specific responses and sensitivities to changes in ocean chemistry are considered to monitor and provide a accurate picture of the effects of acidification on pteropods \citep{Johnson2016,Bednarsek2016CumulativeEffects,Manno2017ReviewPteropodVulnerability,Bednarsek2019MetaAnalysis}. 

% However, current studies assume static ocean chemistry conditions to measure the effect of acidification on shelled pteropods, neglect life stage specific responses to changes in ocean chemistry, as well as the the overall effects on a population level \citep{Bednarsek2019MetaAnalysis}.



%  Overall, exposure to acidification reduces the survival probability of pteropods \citep{Bednarsek2017ApplicationPteropodShell}. The exposure to acidified water conditions decreases shell growth \citep{Comeau2010RepairRates,Bednarsek2014CalcificationDissolution}, calcification \citep{Moya2016NervousSystem}, egg organogenesis \citep{Manno2016EggsAcidification}, downward swimming velocities \citep{Bednarsek2014CalcificationDissolution,Bergan2017SwimmingSinkingSpeeds}, mechanical protection against predators \citep{lalli1989pelagic,Comeau2010Predation}. Additionally, it may result in developmental delays \citep{Thabet2015Lifestages}, shell malformations, or even the complete loss of the aragonite shells \citep{Comeau2010Predation}. The developmental delay and impaired growth rate can reduce the carbon export to deeper layers due to the diminished ballasting effect \citep{Klaas2002Ballast,Bednarsek2014CalcificationDissolution}, and may decrease the flow of energy and carbon across trophic levels \citep{Capuzzo2017AbundanceRecruitment}, which ultimately diminishes global human protein resources \citep{Chassot2010Prot%ein}. Thus, the consequences of exposure of shelled pteropods to acidification has severe implications for biogeochemical cycles and human health.

%possible solution to the missing consideration of life stages/history
% Recent culturing experiments present a detailed life cycle of the temperate \textit{L. retroversa} with ten stages and the developmental timings of key organs/traits \citep{Howes2014Lab,Thabet2015Lifestages} that influence the exposure of pteropods to changes in ocean chemistry.
% The life cycle includes the 2-, 4-, 8-, 16-cell, blastula, gastrula, trochophore, veliger, non-reproductive juveniles, and reproductive adult stage \citep{Thabet2015Lifestages}. 
% The development of key organs include the larval shell (protoconch) and wings (parapodia) after six days from hatching, and three weeks from hatching, respectively \citep{Thabet2015Lifestages}. With the development of the protoconch, early shell dissolution can occur \citep{Thabet2015Lifestages,Johnson2016}, where exposure during early life stages appears to diminish the ability for pteropods to deal with upcoming shell dissolution \citep{Bednarsek2017ExposureHistory}, and is thus crucial for the interpretation of the response of pteropods to changes in ocean chemistry. The development of parapodia marks the onset of the daily upward and downward migration of pteropods across the water column, i.e. diel vertical migration \citep[DVM; ][]{lalli1989pelagic,Mackas2005DVM,Hunt2008TopPredators}, which exposes pteropods to corrosive waters at deeper layers and non-corrosive ones near the surface \citep{Bednarsek2015VerticalDistribution}. Another size or stage dependent trait that should be considered for the interpretation of shell dissolution is the calcification rate or shell repair rate \citep{Comeau2010RepairRates}, which increases with pteropod size \citep{Bednarsek2014CalcificationDissolution}, making juveniles performing DVM relatively more susceptible to shell dissolution relative to the adult organism.



%  Other key traits include the development of wings (parapodia) after three weeks from hatching \citep{Thabet2015Lifestages}, which marks the onset of the upward and downward migration of pteropods across the water column, i.e. diel vertical migration \citep[DVM; ][]{lalli1989pelagic,Mackas2005DVM,Hunt2008TopPredators}.  With the onset of DVM, pteropods can experience extensive shell dissolution at deeper layers and non-corrosive conditions near the surface \citep{Bednarsek2015VerticalDistribution}. Other considerations for the interpretation of shell dissolution is the shell repair rate of pteropods \citep{Comeau2010RepairRates}, which appears to increase with pteropod size \citep{Bednarsek2014CalcificationDissolution}, making juveniles performing DVM relatively more susceptible to shell dissolution relative to the adult organism.









%  the shoaling of corrosive waters in the California Current System \citep{Feely2008Shoaling}, which has a dynamic carbonate chemistry, has been found to result in widespread shell dissolution on individual organisms \citep{Bednarsek2015VerticalDistribution}. However on a population level, a multi-decadal in-situ study of pteropods in the region did not find any significant decrease in their springtime abundance \citep{Ohman2009Multi}. Thus, the population-level approach does not consider indirect or sublethal effects \citep{Stillman2014BenefitsIBM}.



% and few multi-decadal in-situ studies focus on bulk properties, such as the pteropod abundance \citep{Ohman2009Multi} or biomass \citep{Mackas2011Timeseries}, which neglect processes that influence shell structure, morphology or the vertical distribution of pteropods but not the bulk property in question \citep[][]{Ohman2009Multi,Bednarsek2015VerticalDistribution}. Additionally, current studies assume static ocean chemistry conditions to measure the effect of acidification on shelled pteropods \citep{Bednarsek2019MetaAnalysis}, and thus cannot account for life stage specific responses to changes in ocean chemistry \citep{Bednarsek2016CumulativeEffects}, as well as the the overall effects on a population level \citep{Manno2017ReviewPteropodVulnerability,Bednarsek2019MetaAnalysis}. This is likely linked to the uncertainty in the life cycle and growth rates of pteropods mentioned previously. However, as recent studies provide a baseline life cycle and growth rates for shelled pteropods, these limitations might be resolved by simulating pteropod populations as a set of discrete individuals over several generations using an Individual-Based Model \citep[IBM; ][]{DeAngelis2014IBM}.





% The response of individual pteropods to corrosive waters is relatively well studied, we develop in the present study a shelled pteropod IBM, which includes individual-level processes and behaviours, such as changes in acidification sensitivity along the life-cycle of the pteropods \citep{Bednarsek2016CumulativeEffects}, calcification rate with changing shell size \citep{Bednarsek2014CalcificationDissolution}, exposure to acidification due to a size dependent DVM \citep{Maas2012DVM,Bednarsek2015VerticalDistribution}, changes in swimming velocities \citep{Bergan2017SwimmingSinkingSpeeds}, and use the collective response of each individual to determine the sublethal effects of changes in acidification on a population level. In this study we model a pteropod population in the Eastern Boundary Upwelling Systems, specifically the California Current System (CalCS), as a case study for our pteropod IBM, since changes in acidification are likely to disproportionately affect marine ecosystems in these regions \citep{Gruber2011TripleWhammy}. Our IBM (i) reproduces the pteropod abundance signal measured in the CalCS, (ii) quantifies the spatio-temporal variability of shell dissolution in the CalCS throughout the life cycle of shelled pteropods, and (iii) quantifies shell dissolution experienced during DVM and upwelling events.




% However, with recent studies that provide a baseline life cycle \citep{Howes2014Lab,Thabet2015Lifestages} and growth rates \citep{Wang2017Lifecycle} for shelled pteropods
% \textbf{Part on IBM here as proposed solution/tool}
% However, studies have provided a baseline life cycle \citep{Howes2014Lab,Thabet2015Lifestages} and growth rates \citep{Wang2017Lifecycle} for shelled pteropods, as well as a description of key elements such as the upward \citep{Chang2012SwimmingSpeedSize,Murphy2016UpwardSwimming} and downward swimming \citep{Bergan2017SwimmingSinkingSpeeds} speeds of pteropods, their calcification \citep{Comeau2010RepairRates} and dissolution rates as a function of size and ocean chemistry \citep{Bednarsek2014CalcificationDissolution}, or their DVM behavior \citep{Bednarsek2015VerticalDistribution,Bednarsek2017ExposureHistory}.

% The combination of these observations, allows us to simulate the pteropod populations over several generations as a set of discrete individuals, i.e. using an Individual-Based Model \citep[IBM; ][]{DeAngelis2014IBM}, which includes the life stage specific response to acidification, and its interplay with life history, behavioral, physiological, and morphological traits. IBMs are a widely used tool to infer the impacts and responses of a changing environment on populations or communities on a regional or global scale and include individual-level processes and behaviours \citep{DeAngelis2014IBM}. The organism chosen for the IBM can vary between a wide group \citep[e.g. phytoplankton; ][]{Clark2011IBMAdaptations} or a specific species \citep[e.g. Calanus finmarchicus, three-spined stickleback; ][]{Miller1998CalanusIBM,Mintram2018IBM_Stickleback}. Our pteropod IBM includes individual-level processes and behaviours, such as changes in acidification sensitivity along the life-cycle of the pteropods \citep{Bednarsek2016CumulativeEffects}, calcification rate with changing shell size \citep{Bednarsek2014CalcificationDissolution}, exposure to acidification due to DVM with increasing size \citep{Maas2012DVM,Bednarsek2015VerticalDistribution}, as well as changes in their abundance or shell dissolution and size across several generations. 







% However, most studies focus on the response of individual pteropods to acidification on shorter timescales \citep[less than one month; ][]{Manno2017ReviewPteropodVulnerability}, and the few multi-decadal in-situ studies focus on bulk properties (e.g. abundance) and neglect processes that might not significantly change the bulk property \citep[e.g. shell dissolution; ][]{Ohman2009Multi}. Additionally, current studies assume static ocean chemistry conditions to measure the effect of acidification on shelled pteropods, neglect life-stage specific responses to changes in ocean chemistry, as well as the the overall effects on a population level \citep{Bednarsek2019MetaAnalysis}. Thus, in order to monitor and provide a clear picture of the effects of acidification using pteropod shell dissolution, the spatio-temporal characterization of the shell dissolution \citep{Johnson2016} needs to be considered in context with the life-cycle of shelled pteropods over several generations \citep{Manno2017ReviewPteropodVulnerability}, and their life-stage specific responses and sensitivities \citep{Bednarsek2016CumulativeEffects}.


% In sum, studies have successfully documented the life-cycle and life-stages of shelled pteropods \citep{Howes2014Lab,Thabet2015Lifestages}, the upward \citep{Chang2012SwimmingSpeedSize,Murphy2016UpwardSwimming} and downward swimming \citep{Bergan2017SwimmingSinkingSpeeds} speeds, the calcification \citep{Comeau2010RepairRates} and dissolution rates as a function of size and aragonite saturation states \citep{Bednarsek2014CalcificationDissolution}, or have modeled pteropod population level longevity and growth rates \citep{Wang2017Lifecycle}, and DVM behavior \citep{Bednarsek2017ExposureHistory}. The combination of these observations, allows us to simulate the pteropod populations over several generations as a set of discrete individuals, i.e. using an Individual-Based Model \citep[IBM; ][]{DeAngelis2014IBM}, which includes the life stage specific response to acidification, and its interplay with life history, behavioral, physiological, and morphological traits. IBMs are a widely used tool to infer the impacts and responses of a changing environment on populations or communities on a regional or global scale and include individual-level processes and behaviours \citep{DeAngelis2014IBM}. The organism chosen for the IBM can vary between a wide group \citep[e.g. phytoplankton; ][]{Clark2011IBMAdaptations} or a specific species \citep[e.g. Calanus finmarchicus, three-spined stickleback; ][]{Miller1998CalanusIBM,Mintram2018IBM_Stickleback}. Our pteropod IBM includes individual-level processes and behaviours, such as changes in acidification sensitivity along the life-cycle of the pteropods \citep{Bednarsek2016CumulativeEffects}, calcification rate with changing shell size \citep{Bednarsek2014CalcificationDissolution}, exposure to acidification due to DVM with increasing size \citep{Maas2012DVM,Bednarsek2015VerticalDistribution}, as well as changes in their abundance or shell dissolution and size across several generations. 

% In this study we model a pteropod population in the Eastern Boundary Upwelling Systems, specifically the California Current System (CalCS), as a case study for our pteropod IBM, since changes in acidification are likely to disproportionately affect marine ecosystems in these regions \citep{Gruber2011TripleWhammy}. In this study we present a new IBM of shelled pteropod population dynamics, which (i) reproduces the pteropod abundance signal measured in the CalCS, (ii) quantifies the spatio-temporal variability of shell dissolution in the CalCS throughout the life cycle of shelled pteropods, and (iii) quantifies shell dissolution experienced during DVM and upwelling events.





% but due to their aragonite shells which makes them particularly sensitive to changes in ocean chemistry \citep[e.g. ocean acidification; ][]{Orr2005Acidification}.  The thin shells made out of the metastable calcium carbonate mineral aragonite \citep{lalli1989pelagic} is approximately $50\%$ more soluble than its calcite counterpart \citep{Mucci1983Metastable}, where a decrease in seawater carbonate concentration and pH due to the marine uptake  of  atmospheric $CO_2$ \citep{Doney2009OtherCO2} leads  to  widespread  pteropod  shell  dissolution,  and  a  decreased survival \citep{Wall_Palmer2013dissolution,Bednarsek2017ApplicationPteropodShell,Manno2017ReviewPteropodVulnerability}.   Pteropods  are  thus  used  as  an indicator  for  the  overall  response  of  marine  ecosystems  to  changes  in  ocean  chemistry  \citep{Bednarsek2017ApplicationPteropodShell,Manno2017ReviewPteropodVulnerability}.


% However, most studies focus on the response of individual pteropods to acidification on shorter timescales \citep[less than one month; ][]{Manno2017ReviewPteropodVulnerability}, and the few multi-decadal in-situ studies focus on bulk properties (e.g. abundance) and neglect processes that might not significantly change the bulk property \citep[e.g. shell dissolution; ][]{Ohman2009Multi}. Additionally, current studies assume static ocean chemistry conditions to measure the effect of acidification on shelled pteropods, neglect life-stage specific responses to changes in ocean chemistry, as well as the the overall effects on a population level \citep{Bednarsek2019MetaAnalysis}. Thus, in order to monitor and provide a clear picture of the effects of acidification using pteropod shell dissolution, the spatio-temporal characterization of the shell dissolution \citep{Johnson2016} needs to be considered in context with the life-cycle of shelled pteropods over several generations \citep{Manno2017ReviewPteropodVulnerability}, and their life-stage specific responses and sensitivities \citep{Bednarsek2016CumulativeEffects}.




% Additionally, a cohort analysis over three years in the temperate North Pacific identified two generations, i.e. a spring and an overwintering generation, of \textit{L. helicina} per year \citep{Wang2017Lifecycle}. The spring generation is spawned by the overwintering generation in spring, it grows to maturity by the end of summer and spawns the overwintering generation \citep{Wang2017Lifecycle}. Similar to the development under laboratory conditions \citep[e.g. ][]{Howes2014Lab,Thabet2015Lifestages}, the spring generation has a longevity of six months, where phytoplankton spring blooms and summer production allow the development from egg to reproductive adult within this time frame \citep{Wang2017Lifecycle}. However, the overwintering generation undergoes minimal to low growth between autumn and spring, which delays their development and extends their longevity to roughly eleven months \citep{Wang2017Lifecycle}.

%sensitivities
% Pteropods are well documented organisms \citep{Bednarsek2017ApplicationPteropodShell}, not only due to their ubiquity, commercial and biogeochemical importance mentioned above, but due to their aragonite shells which makes them particularly sensitive to changes in ocean chemistry \citep[e.g. ocean acidification; ][]{Orr2005Acidification}.  The thin shells made out of the metastable calcium carbonate mineral aragonite \citep{lalli1989pelagic} is approximately $50\%$ more soluble than its calcite counterpart \citep{Mucci1983Metastable}, where a decrease in seawater carbonate concentration and pH due to the marine uptake  of  atmospheric $CO_2$ \citep{Doney2009OtherCO2} leads  to  widespread  pteropod  shell  dissolution,  and  a  decreased survival \citep{Wall_Palmer2013dissolution,Bednarsek2017ApplicationPteropodShell,Manno2017ReviewPteropodVulnerability}.   Pteropods  are  thus  used  as  an indicator  for  the  overall  response  of  marine  ecosystems  to  changes  in  ocean  chemistry  \citep{Bednarsek2017ApplicationPteropodShell,Manno2017ReviewPteropodVulnerability}.   However, most studies focus on the response of individual pteropods to acidification on shorter timescales \citep[less than one month; ][]{Manno2017ReviewPteropodVulnerability}, and the few multi-decadal in-situ studies focus on bulk properties (e.g. abundance) and neglect processes that might not significantly change the bulk property \citep[e.g. shell dissolution; ][]{Ohman2009Multi}. Additionally, current studies assume static ocean chemistry conditions to measure the effect of acidification on shelled pteropods, neglect life-stage specific responses to changes in ocean chemistry, as well as the the overall effects on a population level \citep{Bednarsek2019MetaAnalysis}. Thus, in order to monitor and provide a clear picture of the effects of acidification using pteropod shell dissolution, the spatio-temporal characterization of the shell dissolution \citep{Johnson2016} needs to be considered in context with the life-cycle of shelled pteropods over several generations \citep{Manno2017ReviewPteropodVulnerability}, and their life-stage specific responses and sensitivities \citep{Bednarsek2016CumulativeEffects}.

%Current knowledge and gaps


%Issues/Challenges


%Need of models to quantify these issues


%What we are going to do












 


 





% In addition to the long-term increase in ocean acidification, shelled pteropods are simultaneously exposed to other environmental stressors, such as ocean warming, deoxygenation, or water freshening \citep{Bednarsek2016CumulativeEffects,Maas2012DVM,Manno2012SwimmingBehaviour}. Currently, some exposure studies \citep[e.g. ][]{Comeau2010RepairRates,Lischka2011WarmingAcidificationJuveniles,Lischka2012SynergisticEffects} have shown constrasting effects of multiple stressors on shelled pteropods. This uncertainty might be related to the species-specific adaptations/behaviours such as the diel vertical migration (DVM) or metabolic suppression \citep[DVM; ][]{Maas2012DVM}, life-stage specific sensitivities \citep{Kroeker2013JuvenilesCalcifiers,Manno2017ReviewPteropodVulnerability,Bednarsek2019MetaAnalysis}, or the exposure history of the pteropods \citep{Bednarsek2017ExposureHistory}. However, the combined effects of acidification, warming, and deoxygenation on shelled pteropods are likely cumulative in nature \citep{Bednarsek2016CumulativeEffects}.  For instance, co-occurring warming and acidification has been shown to increase pteropod mortality, shell degradation, and metabolic rate \citep{Lischka2012SynergisticEffects,Gardner2017SouthernOceanPteropods}. Another example is the combination of water freshening and acidification, where only the simultaneous exposure to acidification and reduced salinity decrease the ability of pteropods to swim upwards and increased their mortality \citep{Manno2012SwimmingBehaviour}. An increase in the metabolic rate, for instance due to warming, shell repair \citep{Hoshijima2017MetabolicRates}, or increased upward swimming activity \citep{Manno2012SwimmingBehaviour},  intensifies the pteropod oxygen requirements, which decreases their metabolic suppression strategy to cope with short-term deoxygenation and acidification \citep{Fabry2008MetabolicSuppression}. As the co-occurrence of ocean acidification, warming and deoxygenation is projected to increase globally \citep{Gruber2011TripleWhammy}, we can expect an intensification of readily observed effects of multiple stressors on shelled pteropods and marine ecosystems.





% In sum, studies have successfully documented the life-cycle and life-stages of shelled pteropods \citep{Howes2014Lab,Thabet2015Lifestages}, the upward \citep{Chang2012SwimmingSpeedSize,Murphy2016UpwardSwimming} and downward swimming \citep{Bergan2017SwimmingSinkingSpeeds} speeds, the calcification \citep{Comeau2010RepairRates} and dissolution rates as a function of size and aragonite saturation states \citep{Bednarsek2014CalcificationDissolution}, or have modeled pteropod population level longevity and growth rates \citep{Wang2017Lifecycle}, and DVM behavior \citep{Bednarsek2017ExposureHistory}. The combination of these observations, allows us to simulate the pteropod populations over several generations as a set of discrete individuals, i.e. using an Individual-Based Model \citep[IBM; ][]{DeAngelis2014IBM}, which includes the life stage specific response to acidification, and its interplay with life history, behavioral, physiological, and morphological traits. IBMs are a widely used tool to infer the impacts and responses of a changing environment on populations or communities on a regional or global scale and include individual-level processes and behaviours \citep{DeAngelis2014IBM}. The organism chosen for the IBM can vary between a wide group \citep[e.g. phytoplankton; ][]{Clark2011IBMAdaptations} or a specific species \citep[e.g. Calanus finmarchicus, three-spined stickleback; ][]{Miller1998CalanusIBM,Mintram2018IBM_Stickleback}. Our pteropod IBM includes individual-level processes and behaviours, such as changes in acidification sensitivity along the life-cycle of the pteropods \citep{Bednarsek2016CumulativeEffects}, calcification rate with changing shell size \citep{Bednarsek2014CalcificationDissolution}, exposure to acidification due to DVM with increasing size \citep{Maas2012DVM,Bednarsek2015VerticalDistribution}, as well as changes in their abundance or shell dissolution and size across several generations. 

% In this study we model a pteropod population in the Eastern Boundary Upwelling Systems, specifically the California Current System (CalCS), as a case study for our pteropod IBM, since changes in acidification are likely to disproportionately affect marine ecosystems in these regions \citep{Gruber2011TripleWhammy}. In this study we present a new IBM of shelled pteropod population dynamics, which (i) reproduces the pteropod abundance signal measured in the CalCS, (ii) quantifies the spatio-temporal variability of shell dissolution in the CalCS throughout the life cycle of shelled pteropods, and (iii) quantifies shell dissolution experienced during DVM and upwelling events.






% IBMs are a widely used tool to infer the impacts and responses of a changing environment on populations or communities on a regional or global scale \citep{DeAngelis2014IBM}. The organism chosen for the IBM can vary between a wide group \citep[e.g. phytoplankton; ][]{Clark2011IBMAdaptations} or a specific species \citep[e.g. Calanus finmarchicus, three-spined stickleback; ][]{Miller1998CalanusIBM,Mintram2018IBM_Stickleback}. Due to the species-specific dependence of pteropod growth rates and longevity to environmental conditions  \citep[e.g. temperature, food availability; ][]{Wang2017Lifecycle}, we focus herein on the group of shelled pteropods instead of individual species. This allows us include broad individual-level processes and behaviours \citep{DeAngelis2014IBM}, such as changes in acidification sensitivity along the life-cycle of the pteropods \citep{Bednarsek2016CumulativeEffects}, their calcification rate with changing shell size \citep{Bednarsek2014CalcificationDissolution}, their exposure to acidification due to DVM with increasing size \citep{Maas2012DVM,Bednarsek2015VerticalDistribution}, as well as changes in their abundance or shell dissolution and size across several generations. Additionally, in this study we focus on the pteropod population found in the Eastern Boundary Upwelling Systems, specifically the California Current System (CalCS), as a case study for our pteropod IBM, since changes in acidification (as well as warming and deoxygenation) are likely to disproportionately affect marine ecosystems in these regions \citep{Gruber2011TripleWhammy}. 

% \textbf{Include follwing points:}
% \begin{itemize}
%     \item Current pteropod models with their advantages and limitations
%     \item What can we improve with this model?
%     \item Open questions in pteropod research and modelling
%     \item Key questions of this study
%     \item Clearly state hypotheses/aim of study
% \end{itemize}



% Due to the dependence of pteropod growth rates and longevity on temperature \citep{Wang2017Lifecycle}, 



% As done for other marine organisms, e.g. marine phytoplankton \citep{Clark2011IBMAdaptations}, three-spined stickleback \citep{Mintram2018IBM_Stickleback}, Calanus finmarchicus \citep{Miller1998CalanusIBM}, or the Medfly \citep{Lux2018IBMAplication}, we present a shelled pteropod individual model which was parameterized using compiled publish experimental data of pteropod development, behaviour and their response to acidification. 

% are a widely used tool to infer the impacts and response of changing environmental conditions on populations or communities, and more importantly, they allow us to simulate the variation of individuals during 


% , such as the marine phytoplankton \citep{Clark2011IBMAdaptations}, three-spined stickleback \citep{Mintram2018IBM_Stickleback}. Additionally, they  Calanus finmarchicus \citep{Miller1998CalanusIBM}, Medfly \citep{Lux2018IBMAplication}. Despite the uncertainty in the growth rates and life-cycle of the shelled pteropods, they represent an optimal candidate for an IBM. As previously mentioned, shelled pteropods are cosmopolitan, and their geographic and vertical extent have been well-documented. Additionally, shelled pteropods have been shown to accumulate their exposure throughout their life, which with known calcification and dissolution rates allows us to trace....




% Goals of study...







% shell the thin aragonite shells of these pteropods , such as the long-term increased in temperature \citep{Lischka2011WarmingAcidificationJuveniles,Gardner2017SouthernOceanPteropods}, acidification \citep[][]{Bednarsek2017ApplicationPteropodShell,Manno2017ReviewPteropodVulnerability}, deoxygenation \citep[][]{Seibel2016HypoxiaMetabilic,Wishner2018DeoxygenetionThresholds}, or their combination \citep{Lischka2012SynergisticEffects,Bednarsek2016CumulativeEffects}


% of the shelled pteropods, the 
% These organisms are sensitive to changing environmental conditions caused by climate change, such as the long-term increased in temperature, acidification, and deoxygenation (REFS). The effects of rising temperature and declining oxygen levels on the metabolic rate (REFS), shell sizes (REFS), or fecundity (REFS) of pteropods has been documented. However, shelled pteropods are overwhelmingly used to quantify the effects of the increasing trend in marine acidification in laboratory conditions (REFS), as well as in several regions of the world such as the California Current System (REFS), Benguela current (REFS), Southern Ocean (REFS), North Pacific Ocean (REFS), Mediterranean Sea (REFS), among others (REFS). The use of shelled pteropods as indicators for acidification This is especially the case, since shelled pteropods produce a shell made out of the metastable carbonate mineral aragonite (REF). This metastable carbonate is more susceptible to increased acidification, and thus shelled pteropods are considered as early indicators for marine acidification (REFS).

% \textbf{EXTREMELY ROUGH PARAGRAPHS BELOW!}

% Temp sensitivity: Seibel et al., 2007; Comeau et al., 2010; Maas et al., 2011; Seibel et al., 2012)
% Marine acidification leads to the wide-spread dissolution of the shells of pteropods. This shell dissolution has been found to have far-reaching consequences for the pteropods themselves, the carbonate export, as well as for higher trophic levels. The dissolution of pteropod shells leads to the loss of buoyancy (REF), reduced sinking and swimming velocity (Bergan2017), developmental delays (REF), reduced shell sizes (REFS), reduced egg organogenesis (REF), reduced protection against predators (REF), and thus an overall fitness loss of the pteropod population. The effects of acidification on the shell size due to developmental delays or impaired growth rate reduces the carbonate export to deeper layers due to the diminished ballasting effect (bednarvsek2019). Additionally, the reduction in size and metabolic rate caused by acidification decreases the flow of energy and carbon across trophic levels (Capuzzo2017), and thus potentially diminishes global human protein resources (Chassot et al., 2010).

% With the projected increase in global marine acidification it is
% From laboratory experiments, it has been shown that the negative response of pteropods to corrosive waters varies with life-stages, where pteropod larvae and juveniles are affected more by higher acidification compared to eggs and adult pteropods (bednarsek2019?). Additionally, in some regions of the world, pteropods are adapted to cope with changes in acidification  the marine organisms in the EBUS are adapted to cope with changes in acidification, some of them live near their upper habitat threshold \citep{Wishner2018,Bednarsek2014}....
% While the effects of increasing acidification have been quantified, the distinction between long term acidification increases and acidification events has not been done. Shelled pteropods are regularly exposed to large changes in acidification due to their Diurnal Vertical Migration (DVM) \citep{ref}. During DVM, adult pteropods can travel up to $XX$ m and, for instance in the EBUS, they experience large changes in the aragonite saturation state \citep{REF}(THIS NEEDS TO BE INTRODUCED ABOVE). In general, the exposure to a reduced aragonite saturation state leads to shell dissolution in pteropods \citep{bednarsek}, loss of buoyancy \citep{REF}, reduced protection from predators \citep{REF}, developmental delays \citep{REF}, and thus a loss of their overall fitness. Additionally,  the negative response of pteropods to corrosive waters varies with life-stage, where pteropod larvae and juveniles are affected more by higher acidification compared to eggs and adult pteropods \citep{bednarsek}. Thus, acidification extremes can have deleterious effects on key life-stages of pteropods depending on their occurrence timing.


% \begin{itemize}
%     \item Introduction of shelled pteropods as keystone species for biogeochemical processes, and marine food-web.
%     \item Sensitivity of pteropods to climate change (temperature, oxygen, acidification) and using them to quantify changes in ecosystem health due to climate change (long-term trend).
%     \item Transition to global acidification and recorded effects on pteropods
%     \item Knowledge gaps on the effects of acidification on pteropods. Susceptibility of different life-stages, role of DVM, effect on abundance peaks, population size, and CaCO$_3$ content. 
%     \item Importance of filling these gaps with a focus on EBUS, since they represent the most productive regions of Earth, and the disproportionate impact of climate change in these regions.
%     \item Pteropod Individual-based model as a possible approach to fill gaps. Using examples from other IBMs (calanus, forams, phyto, etc)
% \end{itemize}


%  In general, the exposure to a reduced aragonite saturation state leads to shell dissolution in pteropods \citep{bednarsek}, loss of buoyancy \citep{REF}, reduced protection from predators \citep{REF}, developmental delays \citep{REF}, and thus a loss of their overall fitness. Additionally,  the negative response of pteropods to corrosive waters varies with life-stage, where pteropod larvae and juveniles are affected more by higher acidification compared to eggs and adult pteropods \citep{bednarsek}. Thus, acidification extremes can have deleterious effects on key life-stages of pteropods depending on their occurrence timing.












% The increasing acidification trend will disproportionately  affect marine calcifiers in the Eastern Boundary Upwelling Systems (EBUS), which are among the most productive and biogeochemically active regions of the planet \citep{chavez2009}. The uneven effects on marine calcifiers in the EBUS are caused by the combination of the increasing trend in global acidification, and the characteristic and naturally recurring upwelling of corrosive waters in these regions\citep{chan2008,hauri2009}, which results in acidification extreme events, i.e. the occurrence of corrosive waters in places where they are usually not found. While the marine organisms in the EBUS are adapted to cope with changes in acidification, some of them live near their upper habitat threshold \citep{Wishner2018,Bednarsek2014}, where acidification extremes will likely be observed.

% The effects of the increasing trend in marine acidification has been readily quantified for in several regions of the world, such as the California Current System, Benguela current, Southern Ocean, North Pacific Ocean, Mediterranean Sea, OTHER EXAMPLES WITH REFERENCES. These studies use the planktonic shelled pteropods to quantify the effect of ocean acidification on marine calcifiers. These organisms represent an optimal indicator species for the quantification of the effects of acidification, since they are cosmopolitan, are sensitive to changes in acidification, are well documented \citep{bednarsek2017b}, dominate the carbon export among pelagic calcifiers \citep{buithenhuis2019}, and are shown to lack an acclimatization capacity to prolonged acidification exposure \citep{bednarsek2017a}. 

% While the effects of increasing acidification have been quantified, the distinction between long term acidification increases and acidification events has not been done. Shelled pteropods are regularly exposed to large changes in acidification due to their Diurnal Vertical Migration (DVM) \citep{ref}. During DVM, adult pteropods can travel up to $XX$ m and, for instance in the EBUS, they experience large changes in the aragonite saturation state \citep{REF}(THIS NEEDS TO BE INTRODUCED ABOVE). In general, the exposure to a reduced aragonite saturation state leads to shell dissolution in pteropods \citep{bednarsek}, loss of buoyancy \citep{REF}, reduced protection from predators \citep{REF}, developmental delays \citep{REF}, and thus a loss of their overall fitness. Additionally,  the negative response of pteropods to corrosive waters varies with life-stage, where pteropod larvae and juveniles are affected more by higher acidification compared to eggs and adult pteropods \citep{bednarsek}. Thus, acidification extremes can have deleterious effects on key life-stages of pteropods depending on their occurrence timing.


% In this study we aim to quantify the effect of acidification events on the fitness of shelled pteropods in the CalCS. FURTHER DESCRIPTION HERE.

% Portable tomographic PIV measurements of swimming shelled Antarctic pteropods


% Title of the planned paragraphs:
% \begin{itemize}
%     \item The global long-term increase in ocean acidification and general effects on marine ecosystems.
%     \item Pronounced impacts of ocean acidification in Easter Boundary Upwelling Systems through extreme acidification.
%     \item Pteropods as a model organism for quantifying the impacts of acidification on marine ecosystems.
%     \item Are there potential gaps in this approach (e.g. population dynamics, life-stages, and cycle of pteropods) and why are they important? (Focus on different responses depending on life-stage, where the timing and magnitude of acidification exposure/extreme events becomes more important/damaging)
%     \item Goals: \begin{enumerate}
%         \item How has the timing and magnitude of acidification extremes changed since 1979-2016?
%         \item Does the population life-cycle change significantly from year to year?
%         \item What changes can be observed in the pteropod population (e.g. reduced abundance, developmental delay, timing of peak abundance, maximum shell diameter)?
%         \item Is there a potential relation between pteropod population-level bottlenecks and extreme event magnitude and timing?
%     \end{enumerate}
% \end{itemize}




% the combination of the naturally recurring changes in acidification and the global increasing acidification trend

% and are thus adapted to these conditions \citep{chan2008} due to upwelling and to the ongoing global increase in acidification. Additionally, with the projected

% Additionally, organisms living in these regions are adapted tothe high disturbance frequency [Chan et al., 2008],  and as some of these organisms live neartheir upper habitat threshold, a response after an extreme events is likely to be observed [e.g.Wishner et al., 2018; Bednarˇsek et al., 2014].

% These regions, while only covering $XX\%$ of the global surface area, produce around $20\%$ of the global capture fisheries \citep{pauly1995}


% Increases in marine acidification disproportionally affect marine calcifiers in the most productive and biogeochemically active regions of the planet, i.e. in the Eastern Boundary Upwelling Systems (EBUS).




% Increased marine acidification disproportionally affects marine organisms in the Eastern Boundary Upwelling Systems (EBUS). These systems are among the most productive and biogeochemically active regions on the planet \citep{Chavez2009}. They cover $XX\%$ of the global surface area and produce around $20\%$ of the global capture fisheries \citep{pauly1995}.

% The  California Current System (CS) has become more acidic due to the uptake of anthropogenic CO$_2$ by the ocean \citep{feely2008}. The area of low pH in the CS is projected to increase and persist for the future \citep{Gruber2012}. The CS is among the most productive and biogeochemically active regions on the planet \citep{chavez2009}, it provides around $20\%$ of the global capture fisheries \citep{pauly}


% The EBUS areamong  the  most  productive  and  biogeochemically  active  regions  on  the  planet  [Chavez  andMessi ́e, 2009].  They provide around20$\%$
% of the global capture fisheries [Pauly and Christensen,1995].  In analogy to terrestrial case study regions, they (i) are regularly exposed to extremeevents [e.g.  acidification; Hauri et al., 2009],  and (ii) are biodiversity hotspots for lower andhigher trophic levels [Checkley and Barth, 2009], with (iii) high phytoplankton and zooplanktonabundances, which have rapid response times to climatic changes [Paerl et al., 2003; Kavanaughet al., 2016; HAYS et al., 2005].  Additionally, organisms living in these regions are adapted tothe high disturbance frequency [Chan et al., 2008],  and as some of these organisms live neartheir upper habitat threshold, a response after an extreme events is likely to be observed [e.g.Wishner et al., 2018; Bednarˇsek et al., 2014].


%  For in-stance,  the  uptake  of  anthropogenic  CO2by  the  ocean  increased  the  area  affected  by  lowerpH-waters in the California Current System [Feely et al., 2008], and this area is projected toincrease or persist for the future [Gruber et al., 2012].

% The uptake of anthropogenic CO$_2$ by the ocean has increased the area affected by lower pH-waters on both global and local scales \citep{feely2008}.


% Auf der Basis dieser Analysen wissen wir ja, wann und wo wir es mit einem Extremereignis zu tun haben. Damit können wir Organismen vergleichen, die eine erhöhte Exposure hatten zu denjenigen die selten in einem Extremereignis waren. 



% Effekt vor allem durch den langfristigen Trend herkommt, oder ob tatsächlich die Extremereignisse eine überproportionale Rolle spielen
